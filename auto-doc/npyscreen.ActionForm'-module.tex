%
% API Documentation for API Documentation
% Module npyscreen.ActionForm'
%
% Generated by epydoc 3.0beta1
% [Thu Mar 29 14:11:32 2007]
%

%%%%%%%%%%%%%%%%%%%%%%%%%%%%%%%%%%%%%%%%%%%%%%%%%%%%%%%%%%%%%%%%%%%%%%%%%%%
%%                          Module Description                           %%
%%%%%%%%%%%%%%%%%%%%%%%%%%%%%%%%%%%%%%%%%%%%%%%%%%%%%%%%%%%%%%%%%%%%%%%%%%%

    \index{npyscreen \textit{(package)}!npyscreen.ActionForm' \textit{(module)}|(}
\section{Module npyscreen.ActionForm'}

    \label{npyscreen:ActionForm'}

%%%%%%%%%%%%%%%%%%%%%%%%%%%%%%%%%%%%%%%%%%%%%%%%%%%%%%%%%%%%%%%%%%%%%%%%%%%
%%                               Functions                               %%
%%%%%%%%%%%%%%%%%%%%%%%%%%%%%%%%%%%%%%%%%%%%%%%%%%%%%%%%%%%%%%%%%%%%%%%%%%%

  \subsection{Functions}

    \label{npyscreen:ActionForm':testme}
    \index{npyscreen \textit{(package)}!npyscreen.ActionForm' \textit{(module)}!npyscreen.ActionForm'.testme \textit{(function)}}

    \vspace{0.5ex}

    \begin{boxedminipage}{\textwidth}

    \raggedright \textbf{testme}(\textit{sa})

    \end{boxedminipage}


%%%%%%%%%%%%%%%%%%%%%%%%%%%%%%%%%%%%%%%%%%%%%%%%%%%%%%%%%%%%%%%%%%%%%%%%%%%
%%                           Class Description                           %%
%%%%%%%%%%%%%%%%%%%%%%%%%%%%%%%%%%%%%%%%%%%%%%%%%%%%%%%%%%%%%%%%%%%%%%%%%%%

    \index{npyscreen \textit{(package)}!npyscreen.ActionForm' \textit{(module)}!npyscreen.ActionForm'.ActionForm \textit{(class)}|(}
\subsection{Class ActionForm}

    \label{npyscreen:ActionForm':ActionForm}
\begin{tabular}{cccccccccc}
% Line for object, linespec=[False, False, False]
\multicolumn{2}{r}{\settowidth{\BCL}{object}\multirow{2}{\BCL}{object}}
&&
&&
&&
  \\\cline{3-3}
  &&\multicolumn{1}{c|}{}
&&
&&
&&
  \\
% Line for npyscreen.screen\_area.ScreenArea, linespec=[False, False]
\multicolumn{4}{r}{\settowidth{\BCL}{npyscreen.screen\_area.ScreenArea}\multirow{2}{\BCL}{npyscreen.screen\_area.ScreenArea}}
&&
&&
  \\\cline{5-5}
  &&&&\multicolumn{1}{c|}{}
&&
&&
  \\
% Line for object, linespec=[False, True, False]
\multicolumn{2}{r}{\settowidth{\BCL}{object}\multirow{2}{\BCL}{object}}
&&
&&\multicolumn{1}{|c}{}
&&
  \\\cline{3-3}
  &&\multicolumn{1}{c|}{}
&&
&\multicolumn{1}{|c}{}&
&&
  \\
% Line for npyscreen.widget.InputHandler, linespec=[True, False]
\multicolumn{4}{r}{\settowidth{\BCL}{npyscreen.widget.InputHandler}\multirow{2}{\BCL}{npyscreen.widget.InputHandler}}
&&\multicolumn{1}{|c}{}
&&
  \\\cline{5-5}
  &&&&\multicolumn{1}{c|}{}
&\multicolumn{1}{|c}{}&
&&
  \\
% Line for {\textless}UNKNOWN{\textgreater}, linespec=[False]
\multicolumn{6}{r}{\settowidth{\BCL}{{\textless}UNKNOWN{\textgreater}}\multirow{2}{\BCL}{{\textless}UNKNOWN{\textgreater}}}
&&
  \\\cline{7-7}
  &&&&&&\multicolumn{1}{c|}{}
&&
  \\
&&&&&&\multicolumn{2}{l}{\textbf{npyscreen.ActionForm'.ActionForm}}
\end{tabular}

\textbf{Known Subclasses:}
npyscreen.Popup'.ActionPopup,
    npyscreen.FormWithMenus'.ActionFormWithMenus

A form with OK and Cancel buttons.  Users should override the on\_ok and 
on\_cancel methods.


%%%%%%%%%%%%%%%%%%%%%%%%%%%%%%%%%%%%%%%%%%%%%%%%%%%%%%%%%%%%%%%%%%%%%%%%%%%
%%                                Methods                                %%
%%%%%%%%%%%%%%%%%%%%%%%%%%%%%%%%%%%%%%%%%%%%%%%%%%%%%%%%%%%%%%%%%%%%%%%%%%%

  \subsubsection{Methods}

    \label{object:__delattr__}
    \index{object.\_\_delattr\_\_ \textit{(function)}}

    \vspace{0.5ex}

    \begin{boxedminipage}{\textwidth}

    \raggedright \textbf{\_\_delattr\_\_}(\textit{...})

    \vspace{-1.5ex}

    \rule{\textwidth}{0.5\fboxrule}
    x.\_\_delattr\_\_('name') {\textless}=={\textgreater} del x.name

    \vspace{1ex}

    \end{boxedminipage}

    \label{object:__getattribute__}
    \index{object.\_\_getattribute\_\_ \textit{(function)}}

    \vspace{0.5ex}

    \begin{boxedminipage}{\textwidth}

    \raggedright \textbf{\_\_getattribute\_\_}(\textit{...})

    \vspace{-1.5ex}

    \rule{\textwidth}{0.5\fboxrule}
    x.\_\_getattribute\_\_('name') {\textless}=={\textgreater} x.name

    \vspace{1ex}

    \end{boxedminipage}

    \label{object:__hash__}
    \index{object.\_\_hash\_\_ \textit{(function)}}

    \vspace{0.5ex}

    \begin{boxedminipage}{\textwidth}

    \raggedright \textbf{\_\_hash\_\_}(\textit{x})

    \vspace{-1.5ex}

    \rule{\textwidth}{0.5\fboxrule}
    hash(x)

    \vspace{1ex}

    \end{boxedminipage}

    \vspace{0.5ex}

    \begin{boxedminipage}{\textwidth}

    \raggedright \textbf{\_\_init\_\_}(\textit{self}, \textit{name}=\texttt{None}, \textit{framed}=\texttt{True}, \textit{help}=\texttt{None}, *\textit{args}, **\textit{keywords})

    x.\_\_init\_\_(...) initializes x; see x.\_\_class\_\_.\_\_doc\_\_ for 
    signature

    \vspace{1ex}

      Overrides: npyscreen.screen\_area.ScreenArea.\_\_init\_\_

    \end{boxedminipage}

    \label{object:__new__}
    \index{object.\_\_new\_\_ \textit{(function)}}

    \vspace{0.5ex}

    \begin{boxedminipage}{\textwidth}

    \raggedright \textbf{\_\_new\_\_}(\textit{T}, \textit{S}, \textit{...})

      \textbf{Return Value}
      \begin{quote}
\begin{alltt}
a new object with type S, a subtype of T
\end{alltt}

      \end{quote}

    \vspace{1ex}

    \end{boxedminipage}

    \label{object:__reduce__}
    \index{object.\_\_reduce\_\_ \textit{(function)}}

    \vspace{0.5ex}

    \begin{boxedminipage}{\textwidth}

    \raggedright \textbf{\_\_reduce\_\_}(\textit{...})

    \vspace{-1.5ex}

    \rule{\textwidth}{0.5\fboxrule}
    helper for pickle

    \vspace{1ex}

    \end{boxedminipage}

    \label{object:__reduce_ex__}
    \index{object.\_\_reduce\_ex\_\_ \textit{(function)}}

    \vspace{0.5ex}

    \begin{boxedminipage}{\textwidth}

    \raggedright \textbf{\_\_reduce\_ex\_\_}(\textit{...})

    \vspace{-1.5ex}

    \rule{\textwidth}{0.5\fboxrule}
    helper for pickle

    \vspace{1ex}

    \end{boxedminipage}

    \label{object:__repr__}
    \index{object.\_\_repr\_\_ \textit{(function)}}

    \vspace{0.5ex}

    \begin{boxedminipage}{\textwidth}

    \raggedright \textbf{\_\_repr\_\_}(\textit{x})

    \vspace{-1.5ex}

    \rule{\textwidth}{0.5\fboxrule}
    repr(x)

    \vspace{1ex}

    \end{boxedminipage}

    \label{object:__setattr__}
    \index{object.\_\_setattr\_\_ \textit{(function)}}

    \vspace{0.5ex}

    \begin{boxedminipage}{\textwidth}

    \raggedright \textbf{\_\_setattr\_\_}(\textit{...})

    \vspace{-1.5ex}

    \rule{\textwidth}{0.5\fboxrule}
    x.\_\_setattr\_\_('name', value) {\textless}=={\textgreater} x.name = 
    value

    \vspace{1ex}

    \end{boxedminipage}

    \label{object:__str__}
    \index{object.\_\_str\_\_ \textit{(function)}}

    \vspace{0.5ex}

    \begin{boxedminipage}{\textwidth}

    \raggedright \textbf{\_\_str\_\_}(\textit{x})

    \vspace{-1.5ex}

    \rule{\textwidth}{0.5\fboxrule}
    str(x)

    \vspace{1ex}

    \end{boxedminipage}

    \label{npyscreen:Form:Form:add}
    
    \vspace{0.5ex}

    \begin{boxedminipage}{\textwidth}

    \raggedright \textbf{add}(\textit{self}, \textit{widgetClass}, \textit{max\_height}=\texttt{None}, \textit{rely}=\texttt{None}, \textit{relx}=\texttt{None}, *\textit{args}, **\textit{keywords})

    \vspace{-1.5ex}

    \rule{\textwidth}{0.5\fboxrule}
    Add a widget to the form.  The form will do its best to decide on 
    placing, unless you override it. The form of this function is 
    add\_widget(WidgetClass, ....) with any arguments or keywords supplied 
    to the widget. The wigdet will be added to self.\_widgets\_\_

    It is safe to use the return value of this function to keep hold of the
    widget, since that is a weak reference proxy, but it is not safe to 
    keep hold of self.\_widgets\_\_

    \vspace{1ex}

    \end{boxedminipage}

    \label{npyscreen:widget:InputHandler:add_complex_handlers}
    \index{npyscreen \textit{(package)}!npyscreen.widget \textit{(module)}!npyscreen.widget.InputHandler \textit{(class)}!npyscreen.widget.InputHandler.add\_complex\_handlers \textit{(method)}}

    \vspace{0.5ex}

    \begin{boxedminipage}{\textwidth}

    \raggedright \textbf{add\_complex\_handlers}(\textit{self}, \textit{handlers\_list})

    \vspace{-1.5ex}

    \rule{\textwidth}{0.5\fboxrule}
    add complex handlers: format of the list is pairs of (test\_function, 
    callback) sets

    \vspace{1ex}

    \end{boxedminipage}

    \label{npyscreen:widget:InputHandler:add_handlers}
    \index{npyscreen \textit{(package)}!npyscreen.widget \textit{(module)}!npyscreen.widget.InputHandler \textit{(class)}!npyscreen.widget.InputHandler.add\_handlers \textit{(method)}}

    \vspace{0.5ex}

    \begin{boxedminipage}{\textwidth}

    \raggedright \textbf{add\_handlers}(\textit{self}, \textit{handler\_dictionary})

    \vspace{-1.5ex}

    \rule{\textwidth}{0.5\fboxrule}
    Update the dictionary of simple handlers.  Pass in a dictionary with 
    keyname (eg "{\textasciicircum}P" or curses.KEY\_DOWN) as the key, and 
    the function that key should call as the values

    \vspace{1ex}

    \end{boxedminipage}

    \label{npyscreen:Form:Form:add_menu}
    
    \vspace{0.5ex}

    \begin{boxedminipage}{\textwidth}

    \raggedright \textbf{add\_menu}(\textit{self}, \textit{menu}=\texttt{None}, \textit{key}=\texttt{None}, *\textit{args}, **\textit{keywords})

    \vspace{-1.5ex}

    \rule{\textwidth}{0.5\fboxrule}
    DEPRICATED

    \vspace{1ex}

    \end{boxedminipage}

    \label{npyscreen:Form:Form:add}
    
    \vspace{0.5ex}

    \begin{boxedminipage}{\textwidth}

    \raggedright \textbf{add\_widget}(\textit{self}, \textit{widgetClass}, \textit{max\_height}=\texttt{None}, \textit{rely}=\texttt{None}, \textit{relx}=\texttt{None}, *\textit{args}, **\textit{keywords})

    \vspace{-1.5ex}

    \rule{\textwidth}{0.5\fboxrule}
    Add a widget to the form.  The form will do its best to decide on 
    placing, unless you override it. The form of this function is 
    add\_widget(WidgetClass, ....) with any arguments or keywords supplied 
    to the widget. The wigdet will be added to self.\_widgets\_\_

    It is safe to use the return value of this function to keep hold of the
    widget, since that is a weak reference proxy, but it is not safe to 
    keep hold of self.\_widgets\_\_

    \vspace{1ex}

    \end{boxedminipage}

    \label{npyscreen:Form:Form:adjust_widgets}
    
    \vspace{0.5ex}

    \begin{boxedminipage}{\textwidth}

    \raggedright \textbf{adjust\_widgets}(\textit{self})

    \vspace{-1.5ex}

    \rule{\textwidth}{0.5\fboxrule}
    This method can be overloaded by derived classes. It is called when 
    editing any widget, as opposed to the while\_editing() method, which 
    may only be called when moving between widgets.  Since it is called for
    every keypress, and perhaps more, be careful when selecting what should
    be done here.

    \vspace{1ex}

    \end{boxedminipage}

    \label{npyscreen:Form:Form:create}
    
    \vspace{0.5ex}

    \begin{boxedminipage}{\textwidth}

    \raggedright \textbf{create}(\textit{self})

    \vspace{-1.5ex}

    \rule{\textwidth}{0.5\fboxrule}
    Programmers should over-ride this in derived classes, creating widgets 
    here

    \vspace{1ex}

    \end{boxedminipage}

    \label{npyscreen:Form:Form:display}
    
    \vspace{0.5ex}

    \begin{boxedminipage}{\textwidth}

    \raggedright \textbf{display}(\textit{self})

    \end{boxedminipage}

    \label{npyscreen:Form:Form:do_nothing}
    
    \vspace{0.5ex}

    \begin{boxedminipage}{\textwidth}

    \raggedright \textbf{do\_nothing}(\textit{self}, *\textit{args}, **\textit{keywords})

    \end{boxedminipage}

    \label{npyscreen:Form:Form:draw_form}
    
    \vspace{0.5ex}

    \begin{boxedminipage}{\textwidth}

    \raggedright \textbf{draw\_form}(\textit{self})

    \end{boxedminipage}

    \vspace{0.5ex}

    \begin{boxedminipage}{\textwidth}

    \raggedright \textbf{edit}(\textit{self})

    Edit the fields until the user selects the ok button added in the lower
    right corner. Button will be removed when editing finishes

    \vspace{1ex}

      Overrides: npyscreen.Form.Form.edit 	extit{(inherited documentation)}

    \end{boxedminipage}

    \label{npyscreen:ActionForm':ActionForm:find_cancel_button}
    \index{npyscreen \textit{(package)}!npyscreen.ActionForm' \textit{(module)}!npyscreen.ActionForm'.ActionForm \textit{(class)}!npyscreen.ActionForm'.ActionForm.find\_cancel\_button \textit{(method)}}

    \vspace{0.5ex}

    \begin{boxedminipage}{\textwidth}

    \raggedright \textbf{find\_cancel\_button}(\textit{self})

    \end{boxedminipage}

    \label{npyscreen:Form:Form:find_next_editable}
    
    \vspace{0.5ex}

    \begin{boxedminipage}{\textwidth}

    \raggedright \textbf{find\_next\_editable}(\textit{self}, *\textit{args})

    \end{boxedminipage}

    \label{npyscreen:Form:Form:find_previous_editable}
    
    \vspace{0.5ex}

    \begin{boxedminipage}{\textwidth}

    \raggedright \textbf{find\_previous\_editable}(\textit{self}, *\textit{args})

    \end{boxedminipage}

    \label{npyscreen:Form:Form:h_display}
    
    \vspace{0.5ex}

    \begin{boxedminipage}{\textwidth}

    \raggedright \textbf{h\_display}(\textit{self}, \textit{input})

    \end{boxedminipage}

    \label{npyscreen:Form:Form:h_display_help}
    
    \vspace{0.5ex}

    \begin{boxedminipage}{\textwidth}

    \raggedright \textbf{h\_display\_help}(\textit{self}, \textit{input})

    \end{boxedminipage}

    \label{npyscreen:widget:InputHandler:h_exit_down}
    \index{npyscreen \textit{(package)}!npyscreen.widget \textit{(module)}!npyscreen.widget.InputHandler \textit{(class)}!npyscreen.widget.InputHandler.h\_exit\_down \textit{(method)}}

    \vspace{0.5ex}

    \begin{boxedminipage}{\textwidth}

    \raggedright \textbf{h\_exit\_down}(\textit{self}, \textit{input})

    \vspace{-1.5ex}

    \rule{\textwidth}{0.5\fboxrule}
    Called when user leaves the widget to the next widget

    \vspace{1ex}

    \end{boxedminipage}

    \label{npyscreen:widget:InputHandler:h_exit_escape}
    \index{npyscreen \textit{(package)}!npyscreen.widget \textit{(module)}!npyscreen.widget.InputHandler \textit{(class)}!npyscreen.widget.InputHandler.h\_exit\_escape \textit{(method)}}

    \vspace{0.5ex}

    \begin{boxedminipage}{\textwidth}

    \raggedright \textbf{h\_exit\_escape}(\textit{self}, \textit{input})

    \end{boxedminipage}

    \label{npyscreen:widget:InputHandler:h_exit_left}
    \index{npyscreen \textit{(package)}!npyscreen.widget \textit{(module)}!npyscreen.widget.InputHandler \textit{(class)}!npyscreen.widget.InputHandler.h\_exit\_left \textit{(method)}}

    \vspace{0.5ex}

    \begin{boxedminipage}{\textwidth}

    \raggedright \textbf{h\_exit\_left}(\textit{self}, \textit{input})

    \end{boxedminipage}

    \label{npyscreen:widget:InputHandler:h_exit_right}
    \index{npyscreen \textit{(package)}!npyscreen.widget \textit{(module)}!npyscreen.widget.InputHandler \textit{(class)}!npyscreen.widget.InputHandler.h\_exit\_right \textit{(method)}}

    \vspace{0.5ex}

    \begin{boxedminipage}{\textwidth}

    \raggedright \textbf{h\_exit\_right}(\textit{self}, \textit{input})

    \end{boxedminipage}

    \label{npyscreen:widget:InputHandler:h_exit_up}
    \index{npyscreen \textit{(package)}!npyscreen.widget \textit{(module)}!npyscreen.widget.InputHandler \textit{(class)}!npyscreen.widget.InputHandler.h\_exit\_up \textit{(method)}}

    \vspace{0.5ex}

    \begin{boxedminipage}{\textwidth}

    \raggedright \textbf{h\_exit\_up}(\textit{self}, \textit{input})

    \vspace{-1.5ex}

    \rule{\textwidth}{0.5\fboxrule}
    Called when the user leaves the widget to the previous widget

    \vspace{1ex}

    \end{boxedminipage}

    \label{npyscreen:Form:Form:handle_exiting_widgets}
    
    \vspace{0.5ex}

    \begin{boxedminipage}{\textwidth}

    \raggedright \textbf{handle\_exiting\_widgets}(\textit{self}, \textit{condition})

    \end{boxedminipage}

    \label{npyscreen:widget:InputHandler:handle_input}
    \index{npyscreen \textit{(package)}!npyscreen.widget \textit{(module)}!npyscreen.widget.InputHandler \textit{(class)}!npyscreen.widget.InputHandler.handle\_input \textit{(method)}}

    \vspace{0.5ex}

    \begin{boxedminipage}{\textwidth}

    \raggedright \textbf{handle\_input}(\textit{self}, \textit{input})

    \vspace{-1.5ex}

    \rule{\textwidth}{0.5\fboxrule}
    Returns True if input has been dealt with, and no further action needs 
    taking. First attempts to look up a method in self.input\_handers 
    (which is a dictionary), then runs the methods in 
    self.complex\_handlers (if any), which is an array of form (test\_func,
    dispatch\_func). If test\_func(input) returns true, then 
    dispatch\_func(input) is called. Check to see if parent can handle. No 
    further action taken after that point.

    \vspace{1ex}

    \end{boxedminipage}

    \label{npyscreen:Form:Form:menuOfMenus}
    
    \vspace{0.5ex}

    \begin{boxedminipage}{\textwidth}

    \raggedright \textbf{menuOfMenus}(\textit{self}, *\textit{args}, **\textit{keywords})

    \vspace{-1.5ex}

    \rule{\textwidth}{0.5\fboxrule}
    DEPRICATED

    \vspace{1ex}

    \end{boxedminipage}

    \label{npyscreen:ActionForm':ActionForm:on_cancel}
    \index{npyscreen \textit{(package)}!npyscreen.ActionForm' \textit{(module)}!npyscreen.ActionForm'.ActionForm \textit{(class)}!npyscreen.ActionForm'.ActionForm.on\_cancel \textit{(method)}}

    \vspace{0.5ex}

    \begin{boxedminipage}{\textwidth}

    \raggedright \textbf{on\_cancel}(\textit{self})

    \end{boxedminipage}

    \label{npyscreen:ActionForm':ActionForm:on_ok}
    \index{npyscreen \textit{(package)}!npyscreen.ActionForm' \textit{(module)}!npyscreen.ActionForm'.ActionForm \textit{(class)}!npyscreen.ActionForm'.ActionForm.on\_ok \textit{(method)}}

    \vspace{0.5ex}

    \begin{boxedminipage}{\textwidth}

    \raggedright \textbf{on\_ok}(\textit{self})

    \end{boxedminipage}

    \label{npyscreen:Form:Form:on_screen}
    
    \vspace{0.5ex}

    \begin{boxedminipage}{\textwidth}

    \raggedright \textbf{on\_screen}(\textit{self})

    \end{boxedminipage}

    \label{npyscreen:screen_area:ScreenArea:refresh}
    \index{npyscreen \textit{(package)}!npyscreen.screen\_area \textit{(module)}!npyscreen.screen\_area.ScreenArea \textit{(class)}!npyscreen.screen\_area.ScreenArea.refresh \textit{(method)}}

    \vspace{0.5ex}

    \begin{boxedminipage}{\textwidth}

    \raggedright \textbf{refresh}(\textit{self})

    \end{boxedminipage}

    \vspace{0.5ex}

    \begin{boxedminipage}{\textwidth}

    \raggedright \textbf{set\_up\_exit\_condition\_handlers}(\textit{self})

      Overrides: npyscreen.Form.Form.set\_up\_exit\_condition\_handlers

    \end{boxedminipage}

    \vspace{0.5ex}

    \begin{boxedminipage}{\textwidth}

    \raggedright \textbf{set\_up\_handlers}(\textit{self})

    This function should be called somewhere during object initialisation 
    (which all library-defined widgets do). You might like to override this
    in your own definition, but in most cases the add\_handers or 
    add\_complex\_handlers methods are what you want.

    \vspace{1ex}

      Overrides: npyscreen.widget.InputHandler.set\_up\_handlers 	extit{(inherited documentation)}

    \end{boxedminipage}

    \label{npyscreen:Form:Form:useable_space}
    
    \vspace{0.5ex}

    \begin{boxedminipage}{\textwidth}

    \raggedright \textbf{useable\_space}(\textit{self}, \textit{rely}=\texttt{0}, \textit{relx}=\texttt{0})

    \vspace{-1.5ex}

    \rule{\textwidth}{0.5\fboxrule}
    Reports space left on physical screen. Widgets should use 
    widget\_useable\_space instead.

    \vspace{1ex}

    \end{boxedminipage}

    \label{npyscreen:Form:Form:while_editing}
    
    \vspace{0.5ex}

    \begin{boxedminipage}{\textwidth}

    \raggedright \textbf{while\_editing}(\textit{self}, *\textit{args}, **\textit{keywords})

    \vspace{-1.5ex}

    \rule{\textwidth}{0.5\fboxrule}
    This function gets called during the edit loop, on each iteration of 
    the loop.  It does nothing: it is here to make customising the loop as 
    easy as overriding this function. A proxy to the currently selected 
    widget is passed to the function.

    \vspace{1ex}

    \end{boxedminipage}

    \label{npyscreen:screen_area:ScreenArea:widget_useable_space}
    \index{npyscreen \textit{(package)}!npyscreen.screen\_area \textit{(module)}!npyscreen.screen\_area.ScreenArea \textit{(class)}!npyscreen.screen\_area.ScreenArea.widget\_useable\_space \textit{(method)}}

    \vspace{0.5ex}

    \begin{boxedminipage}{\textwidth}

    \raggedright \textbf{widget\_useable\_space}(\textit{self}, \textit{rely}=\texttt{0}, \textit{relx}=\texttt{0})

    \end{boxedminipage}


%%%%%%%%%%%%%%%%%%%%%%%%%%%%%%%%%%%%%%%%%%%%%%%%%%%%%%%%%%%%%%%%%%%%%%%%%%%
%%                              Properties                               %%
%%%%%%%%%%%%%%%%%%%%%%%%%%%%%%%%%%%%%%%%%%%%%%%%%%%%%%%%%%%%%%%%%%%%%%%%%%%

  \subsubsection{Properties}

\begin{longtable}{|p{.30\textwidth}|p{.62\textwidth}|l}
\cline{1-2}
\cline{1-2} \centering \textbf{Name} & \centering \textbf{Description}& \\
\cline{1-2}
\endhead\cline{1-2}\multicolumn{3}{r}{\small\textit{continued on next page}}\\\endfoot\cline{1-2}
\endlastfoot\raggedright \_\-\_\-c\-l\-a\-s\-s\-\_\-\_\- & \raggedright \textbf{Value:} 
{\tt {\textless}attribute '\_\_class\_\_' of 'object' objects{\textgreater}}&\\
\cline{1-2}
\end{longtable}


%%%%%%%%%%%%%%%%%%%%%%%%%%%%%%%%%%%%%%%%%%%%%%%%%%%%%%%%%%%%%%%%%%%%%%%%%%%
%%                            Class Variables                            %%
%%%%%%%%%%%%%%%%%%%%%%%%%%%%%%%%%%%%%%%%%%%%%%%%%%%%%%%%%%%%%%%%%%%%%%%%%%%

  \subsubsection{Class Variables}

\begin{longtable}{|p{.30\textwidth}|p{.62\textwidth}|l}
\cline{1-2}
\cline{1-2} \centering \textbf{Name} & \centering \textbf{Description}& \\
\cline{1-2}
\endhead\cline{1-2}\multicolumn{3}{r}{\small\textit{continued on next page}}\\\endfoot\cline{1-2}
\endlastfoot\raggedright C\-A\-N\-C\-E\-L\-\_\-B\-U\-T\-T\-O\-N\-\_\-B\-R\-\_\-O\-F\-F\-S\-E\-T\- & \raggedright \textbf{Value:} 
{\tt \texttt{(}2\texttt{, }12\texttt{)}}&\\
\cline{1-2}
\raggedright D\-E\-F\-A\-U\-L\-T\-\_\-X\-\_\-O\-F\-F\-S\-E\-T\- & \raggedright \textbf{Value:} 
{\tt 2}&\\
\cline{1-2}
\raggedright O\-K\-\_\-B\-U\-T\-T\-O\-N\-\_\-B\-R\-\_\-O\-F\-F\-S\-E\-T\- & \raggedright \textbf{Value:} 
{\tt \texttt{(}2\texttt{, }6\texttt{)}}&\\
\cline{1-2}
\end{longtable}

    \index{npyscreen \textit{(package)}!npyscreen.ActionForm' \textit{(module)}!npyscreen.ActionForm'.ActionForm \textit{(class)}|)}
    \index{npyscreen \textit{(package)}!npyscreen.ActionForm' \textit{(module)}|)}
